
\documentclass[]{article}
\usepackage{amsmath,amssymb,color}

%opening
\title{Analysis Spatial pattern}
\author{Lorette Noiret}

\begin{document}

\maketitle
\section{Study goals}
\subsection{Aims}
Understand the spatio-temporal pattern of gene expression and the link this pattern with the phenotypes (pattern of lamination...). \\
In a first step, we are focusing on each time point separately. This step should also help to identify regions that will be used for single-cell analyses.
\begin{itemize}
	\item Identifiation of the regions.
	\begin{itemize}
		\item How many regions can we identify? 
		\item Should we identify the regions manually (contouring) or automatically?
		\item Should we identify the region based on gene expressions (descritptive analysis, clusetering) or based the phenotypic pattern (PLS?)?
		\item is it necessary to apply a mask (defined manually) to find those regions (e.g. exclusion cuticule)? Yohanns also suggested to create a mask automatically (by selecting x number of pixels around the `zero area')
		\item should we average the expression of a specific gene in several animals or only consider one animal at a time?
	\end{itemize}
	\item Analyse the pattern of some key gene expression. 
		\begin{itemize}
		\item which genes are co-expressed together in each region? Clustering is useful to better understand the pattern even if we do not use the results for the single-cell analysis. 
		\item do the pattern of gene expression (spatial organization of cluster) correlate with the phenotypic pattern?
	\end{itemize}
\end{itemize}
Other ideas:\\
Compare the gene expression profiles with the fly  of different size. \\



\section{Step1 - Spatial analysis at 12apf}

\subsection{Data}
Genes express in the notum at 12apf.\\
\newline
Source directory: `Patt\_CellProp\_Rescaled\_2017\_sept\_06th\_test\_raw\_data'.\\
\newline
Initial dataset:
\begin{itemize} 
\item Yohanns and Maria selected 10 genes of interest (ara, BH1, bi, esg, eyg, hth, pct, sd, Sr, Ush) measured in 10 individuals. {\color{blue} Better understand the limitations associated with data acquisation : expression, max projection, stiching...}
\item Some phenotypic map are also available (Delamination, proliferation,TissueDeform,TissueStress)
\item a file contains the macrocheataetae
\end{itemize}\\
Comments and challenges associated with these data:
\begin{itemize}
	\item The gene barH1 (bh1) was measured on a smaller notum, so comparisons we should perform the clustering with and without it and compare the results
	\item Nothing tell us that those genes are the one explaining the spatial phenotypic organization (at that time point). Maybe a PLS before clustering could be useful (more than a PCA)
	\item Should we pool the data from several animals?
	\item Image includes the growing cuticule, should we exclude it (by applying a mask delimited by Maria or an automatic approach) 	 
\end{itemize}
%{\color{blue} Some genes initially mentioned are not inc}
\subsection{Statistical Analysis}
\subsubsection{Workflow}
\begin{enumerate}
	\item read image
	\item Apply mask (only if UseMask==1)
	\item Normalize each image on 0-1 scale (divide all the pixel by the maximum intensity value). {\color{blue} Impact normalization?}
	\item vectorize image
	\item {TO DO: impact if we do a PCA before hand?}
	\item remove unwanted (if pixel=0 on all image then it is a background point and I can exclude it to make the code faster)
	\item k-mean 2 to 10 clusters (max iteration 1000). 
	\begin{itemize}
		\item Use 3 to 5 replicates to insure that we find a global minimum.
		\item Assess quality clustering: silhouette (how well each object lies within cluster), but not doable of full-size image. BIC but based on log-likelihood (gaussian assumption seems wrong). Wilks statistics 
		
	\end{itemize} 
\end{enumerate}
\subsubsection{Smoothing}
The images have taken in high resolution and the nucleus are apparent (black area). We should try to do some interpolation before clustering?
\begin{itemize}
	\item Smoothing data: using gaussian filter or another one 
	\item resize: decrease the image size with a filter (resizem). Advantage: faster computation, can compute silhouette. 
	\item kriging: spatial interpolation, can take into account anisotropie and discontinuity {\color{blue} TO DO, available in Matlab?? }
\end{itemize}(filtering, resize or kriging)?\\
Challenges: how to choose the optimal parameters (sigma/distance for gaussian filter, scaling and filter for resizing)?\\
Smoothing with a distance of 15 pixels remove completely the discountinuous aspects, but also blur the small patterns (becoming non-apparent when create the clusters).
\subsubsection{Automatic detection of cluster (methods)}
Different approach can be tested:
\begin{itemize}
	\item Hierarchical approach (AHC): tried it but matlab memory bugged. Maybe I did something wrong or should implement it in Python?
	\item Kmeans
	{\color{blue} I tried euclidian distance, but could try other metric as well. Ask bioinformatician if some metric works better with gene expression }
	\item Kmedoid: interest if we look at the expression qualitatively (expressed / not expressed). Advantage: center of class is a datapoint (use L1 metric). {\color{blue} TO DO, available in matlab}
	\item {\color{blue}Quid of spatial clustering. Some approach must exist, check litterature}
\end{itemize}
\paragraph{kmeans}

To identify the regions, we classify the 10 genes expression map using a k-means algotithm (from 2 to 10 clusters). \\
How to determine the optimal number of cluster?
\begin{itemize}
	\item Compare pattern with the one defined by Maria
	\item Look at some automatic features:
		\begin{itemize}
			\item silhouette
			\item Wilks statistic, variance within cluster, distance between centroids
			\item Stability : how robust a clustering solution is under pertubation or sub-sampling
			\item 
		\end{itemize} 
\end{itemize} 
Other technical aspects
	\begin{itemize}
		\item Number of repeats: kmeans with n groups performed several times to limit the impact of centroids inititialization. I took 3, maybe should go up to 5.
		\item Choice metric. Only tested euclidian so far
		\item Choice color for clusters. Choice of color makes a difference in term of vizualization and apparent quality of clusters 
	\end{itemize}
\noindent {\bf Open questions (to discuss):}
\begin{itemize}
\item Data are not continuous (cells border). Should we smooth the image first?
\item Should we consider another algorithm? I initially tried hierarchical cluster analysis, but the size was to large, and Matlab crashes. Is there a better way to do it (another sotware? another algorithm?)?
\end{itemize} 
\subsubsection{Other points}
{\color{blue} Does a preliminary PCA change the results?
Rather than PCA, PLS could help?}\\

\subsubsection{Classification approach}

\subsection{Results}
\subsubsection{Descriptive analysis}
Visually description of the pattern of each gene and phenotype.\\

\begin{tabular}{@{}p{0.15\linewidth}p{0.3 \linewidth}p{0.65\linewidth}}%{l|c|c}
name&some functions&comments\\ \hline
ara (araucan)&TF, notum cell fate specification&	expressed latterally (except 2 zones ({\color{blue}wings or another phenotype later?}), midline +/- empty, a few intense spots\\ 
%
bh1 (BarH1)&TF, chaeta morphogenesis; compound eye photoreceptor cell differentiation& Fly is much smaller $\rightarrow$ {\color{red} problem?}, opposition posterior/anterior, only expressed in region clode to neck + few spot close to scutellum\\
%
bi (Bifid)& TF, development of several tissues such as brain, eyes and wings&pattern both vertical and horizontal. More expressed in scutellum, 3 empty spots in vertical middle, problem stitching?\\
%
esg (Escargot)& TF, maintenance of cell number&scutellum except 2 spots and along horizontal axis ({\color{blue} physio?})\\
%
eyg (eyegone)& TF (transcriptional repressor), notum development&not in scutelum, pattern not uniform in scutum (vertical pattern),{\color{blue} true or artefact?}  \\
%
hth (homothorax)& TF, phenotypes manifest in adult abdominal segment, regulation of cell fate commitment; macromolecule localization, formation of anatomical boundary&{\color{blue}signal seems noisy},scutellum, 2 patch in lateral scutum\\
ptc (patched)& hedgehog receptor activity, cell morphogenesis involved in differentiation; columnar/cuboidal epithelial cell differentiation & expressed on the two posterior/anterior lines, ({\color{blue}neck}\\
%
sd (scalloped)& TF,tissue morphogenesis; regulation cell communication, regul multicellular organismal development, stem cell proliferation&essential posterior line\\
%
Sr (stripe)&nucleic acid binding, epithelial cell migration, ectoderm development, determination of muscle attachment site&parallel horizontal strip in scutum\\
%
Ush (u-shaped)& TF binding,leading edge cell differentiation, pattern, epithelial cell fate commitment specification process&everywhere except lateral border and 2 spots in scutellum\\
%
Delamination &&scutellum and horizontal midline \\
Proliferation&&scutellum, horizontal midline + parallel stripe in scutum\\
Deformation  &&\\
Stress       &&\\
\end{tabular}
\\
\newline
TF = transcription factor\\
Comments:
\begin{itemize} 
\item ask Boris to describe Deformation and stress\\
\item No phenotypic region correspond to the empty lateral spots and 2 spot in scutellum.
\end{itemize}
\subsubsection{Clustering analysis}
\paragraph{Using raw data or the mask?}
Le noir et 20 en plus 
\paragraph{Manual versus automatic}
%faire : PCA
\end{document}
